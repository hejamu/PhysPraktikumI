\chapter{Formeln}
Wechselströme können mithilfe imaginärer Exponentialfunktionen beschrieben werden:\\
\begin{equation}
I=I_0 \cdot \exp(i \cdot \omega \cdot t)
\end{equation}
$I_0$ entspricht der Amplitude, $\omega$ der Kreisfrequenz.\\
\\
Für eine Spule gilt:\\
\begin{equation}
U=L \cdot I
\end{equation}
\begin{equation}
Z_L= \frac{U}{I} = i \cdot \omega \cdot L
\end{equation}
\\
Für einen Kondensator gilt:\\
\begin{equation}
I = C\cdot\dot{U}
\end{equation}
 \begin{equation}
 Z_C = \frac{U}{I} = -\frac{i}{\omega \cdot C}
 \end{equation}
C entspricht der Kapazität.\\
\\

Nach einem Brückenabgleich gilt in der Wheatstonschen Brückenschaltung:
\begin{equation}
	\frac{Z_x}{Z_3} = \frac{Z_2}{Z_4}
\end{equation}

Damit gilt in der Reihenschaltung
\begin{equation}
	Z_x = Z_2 = R_n + \frac{1}{i\omega C} = 
R_n - \frac{1}{\omega \cdot C} i 
\end{equation}
Für den Real- und Imaginärteil gilt dann 
\begin{align}
	Re(Z_x) &= R_n \\
	Im(Z_x) &= - \frac{1}{2\pi f C}
\end{align}

Für die Parallelschaltung gilt dann
\begin{align}
 Z_x = Z_2Z_3 \frac{1}{Z_4} &= {\frac{R_2R_3}{R_n} +  R_2R_3 \omega C i} 
\end{align}
Für den Real- und Imaginärteil gilt dann 
\begin{align}
	Re(Z_x) &=  \frac{R_2R_3}{R_n}\\
	Im(Z_x) &= R_2R_3 \omega C 
\end{align}

Für die Auswertung wird zusätzlich der Betrag
\begin{equation}
	|Z_x| = \sqrt{Re(Z_x)^2 + Im(Z_x)^2}
\end{equation}
und die Phase 
\begin{equation}
	\phi =  \arctan\left(\frac{Im(Z)}{Re(Z)}\right)
\end{equation}
	\pagebreak