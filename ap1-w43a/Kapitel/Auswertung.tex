\chapter{Auswertung}
    	\section{Bestimmung der Adiabatenexponenten}
         Der Adiabatenexponent $\gamma$ wird mit der Formel \eqref{gamma} bestimmt. Dabei wurde das Volumen des Gaseinlassventils bereits in $V_0$ einbezogen. Berechnet für den ersten Wert:
         
         \begin{align*}
         	\gamma_{\text{Luft}} &= \frac{4\pi^2 m V_0f_0^2}{A^2 p_0} \\
            	&= \frac{4\pi^2 0,0088781 kg \cdot 0,0000911779 m^3 \cdot (13,4 Hz)^2}{(0,000201062 m^2)^2 \cdot 103100 Pa} \\
                &= 1,38
         \end{align*}
        
        
       
         
         Die Tabellen 5.1, 5.2 und 5.3 enthalten die analog errechneten Werte:
         \begin{center}
         
         \begin{tabular}{c|c}
$l ~ [\text{m}]$ & $\gamma ~ [1]$    \\ \hline
0,45 & 1,38 \\
0,40 & 1,36 \\
0,35 & 1,44 \\
0,30 & 1,38 \\
0,25 & 1,36 \\
0,20 & 1,34 \\
0,15 & 1,35 \\
0,10 & 1,31 
\end{tabular}
\captionof{table}[]{Messungen des Adiabatenexponenten mit Luft als Kolbenfüllung.}
\end{center}
\begin{minipage}{\textwidth}
\begin{center}
         \begin{tabular}{c|c}
$l ~ [\text{m}]$ & $\gamma ~ [1]$    \\ \hline
 0,45	& 1,38 \\
 0,40	& 1,34 \\
 0,35	& 1,31 \\
 0,30	& 1,28 \\
 0,25	& 1,30 \\
 0,20	& 1,28 \\
 0,15	& 1,30 \\ 
 0,10	& 1,23
\end{tabular}
\captionof{table}[]{Messungen des Adiabatenexponenten mit Kohlenstoffdioxid als Kolbenfüllung.}         
\end{center}
\end{minipage}
\begin{center}
\ \\         
\begin{tabular}{c|c}
$l ~ [\text{m}]$ & $\gamma ~ [1]$    \\ \hline
0,45   & 1,66 \\
0,40    & 1,60 \\
0,35   & 1,65 \\
0,30    & 1,55 \\
0,24   & 1,53 \\
0,19   & 1,41 \\
0,14   & 2,00 \\
0,095  & 1,41 
\end{tabular}
\captionof{table}[]{Messungen des Adiabatenexponenten mit Argon als Kolbenfüllung.}
\end{center}
\ \\
        \section{Vergleich mit Literaturwerten}
        	Für den Mittelwert wird jeweils der letzte Messwert ausgelassen, da sich eine Messung ab einer niedrigen Höhe von $l = 10  ~\si{\centi\metre}$ als sehr schwierig erwiesen hat. Das könnte daran liegen, dass die Differenz zwischen $V(l)$, dem Volumen in Abhängigkeit von der Position, und $V_0$, dem Volumen in der Ruhelage, kleiner wird.
            
            \begin{align*}
             \bar \gamma_{\text{Luft}} = \frac{\sum_{i=1}^7 \gamma_{\text{Luft}}^{(i)}}{7} = 1,37 
            \end{align*}
            Analog für Kohlenstoffdioxid und Argon
            \begin{align*}
             \bar \gamma_{\text{CO2}} &= 1,31 \\
             \bar \gamma_{\text{Argon}} &= 1,63
            \end{align*}
\ \\
    		Die Literaturwerte für die verwendeten Gase sind in Tabelle 4.4 aufgelistet 
            \begin{center}
            	\begin{tabular}{c|c}
            		Medium & $\gamma ~[1]$ \\ \hline
                    Luft (N$_2$) & 1,40 \\
                    CO$_2$ &  1,29\\
                    Argon & 1,67\\
            	\end{tabular}
                \captionof{table}[]{Literaturwerte des Adiabatenexponenten der Gase bei 20 Grad Celsius und Normaldruck \cite{art:test}}
            \end{center}

            Die Mittelwerte der Messung liegen alle recht nahe an den angegebenen Literaturwerten.

        
        
	\pagebreak