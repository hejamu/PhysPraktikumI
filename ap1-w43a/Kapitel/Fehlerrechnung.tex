\chapter{Fehlerrechnung}
	Um das Ventilvolumen in die Formel einzubringen, wird $\gamma$ in Abhängigkeit der Länge $l$, statt der Fläche $A$ und dem Volumen $V$ geschrieben.
      \begin{align*}
     		\gamma = \frac{16 \pi \cdot m \cdot l \cdot f_0^2}{d^2 \cdot p_0}
      \end{align*}
      \begin{align*}
     		\Delta\gamma_{\text{Luft}} &= 
            \left| \frac{16\pi l f_0^2}{d^2 p_0}\right| \Delta m + 
            \left| \frac{16\pi m f_0^2}{d^2 p_0}\right| \Delta l + 
            \left| \frac{32\pi m l f_0}{d^2 p_0}\right| \Delta f_0 + 
            \left| -\frac{32 \pi m l f_0^2}{d^3 p_0}\right| \Delta d \\&+ 
            \left| -\frac{16 \pi m l f_0^2}{d^2 p_0^2}\right| \Delta p_0 \\&=
            \left| \frac{16\pi (0,45m + 0,001m) (13,4Hz)^2}{(0,016m)^2 (103100Pa)}\right| (0,0000001kg)\\ &+ 
            \left| \frac{16\pi (0,0088781kg) (13,4Hz)^2}{(0,016m)^2 (103100Pa)}\right| (0,001m) \\ &+ 
            \left| \frac{32\pi (0,0088781kg) (0,45m + 0,001m) (13,4Hz)}{(0,016m)^2 (103100Pa)}\right| (0,2Hz) \\ &+ 
            \left| -\frac{32 \pi (0,0088781kg) (0,45m + 0,001m) (13,4Hz)^2}{(0,016m)^3 (103100Pa)}\right| (0,00001m)\\&+ 
            \left| -\frac{16 \pi (0,0088781kg) (0,45m + 0,001m) (13,4Hz)^2}{(0,016m)^2 (103100Pa)^2}\right| (100Pa) \\ &= 0,048
      \end{align*} \\
      
      Analog für die weiteren Fehler
      \begin{center}
      \begin{tabular}{c|c}
$l ~[\text{m}]$ & $\Delta\gamma ~[1]$    \\ \hline
0,45         & 0,048 \\
0,40          & 0,045\\
0,35         & 0,045 \\
0,30          & 0,042 \\
0,25         & 0,039 \\
0,20          & 0,037 \\
0,15         & 0,036 \\
0,10          & 0,035	
		\end{tabular}
	\end{center}
      
\captionof{table}[]{Fehler für die Messung mit Luft als Kolbenfüllung.}
\begin{center}
      \begin{tabular}{c|c}
$l ~[\text{m}]$ & $\Delta\gamma ~[1]$    \\ \hline
0,45 & 0,048 \\
0,40  & 0,045 \\
0,35 & 0,042 \\
0,30  & 0,040 \\
0,25 & 0,038 \\
0,20  & 0,036 \\
0,15 & 0,035 \\
0,10  & 0,034
		\end{tabular}
	\end{center}
      
\captionof{table}[]{Fehler für die Messung mit Kohlenstoffdioxid als Kolbenfüllung.}

\begin{center}
      \begin{tabular}{c|c}
$l ~ [\text{m}]$ & $\Delta\gamma ~[1]$    \\ \hline
0,45   & 0,053 \\
0,40    & 0,050 \\
0,35   & 0,048 \\
0,30    & 0,045 \\
0,24   & 0,042 \\
0,19   & 0,038 \\
0,14   & 0,047 \\
0,095  & 0,038 
		\end{tabular}
	\end{center}
      
\captionof{table}[]{Fehler für die Messung mit Argon als Kolbenfüllung.}
\ \\
Es soll nur ein Wert für den Fehler angegeben werden. Anstatt wie bei den Messwerten zu mitteln, wird hier jedoch der größte Fehler angenommen, damit der Fehlerwert nicht zu gering ausfällt.

\begin{align*}
 	\Delta \gamma_{\text{Luft}} & = 0,048\\
    \Delta \gamma_{\text{CO}_2} & = 0,048\\
    \Delta \gamma_{\text{Argon}}& = 0,053
\end{align*}

     	\section{Mögliche Abweichungen}
        	Im Versuch wird angenommen, dass die Erregerfrequenz beim Maximum der Schwingung der Eigenfrequenz des ungedämpften Systems entspricht. Durch Dämpfung fällt die Frequenz jedoch geringer aus, was sich durch den Zusammenhang $\gamma \propto f^2$ stark auf das Ergebnis auswirkt. Der Wert für den gemessenen Adiabatenexponent liegt also unter dem tatsächlichen. Außerdem wird nicht mit einem isolierten System gemessen, sodass der Vorgang nicht vollständig adiabatisch verläuft. Reibung verursacht ebenfalls einen Fehler.
	\pagebreak