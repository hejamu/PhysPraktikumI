\documentclass[12pt,a4paper,]{scrreprt}
\usepackage[ngerman]{babel}
\usepackage[onehalfspacing]{setspace}
\usepackage[utf8]{inputenc}

\usepackage{graphicx,array,gnuplottex,siunitx,multicol,capt-of,amsmath,ulem,amsthm}
\let\phi\varphi
\setkomafont{chapter}{\fontsize{20bp}{22.2bp}\selectfont\bfseries}

\setkomafont{chapter}{\fontsize{14bp}{18.8bp}\selectfont\bfseries}
\setkomafont{section}{\fontsize{12bp}{14.4bp}\selectfont\bfseries}

\renewcommand{\chapterheadstartvskip}{\vspace*{-1\topskip}}
\renewcommand{\chapterheadendvskip}{\vspace*{0.8\topskip}}
%---------------------------------------------------------------------------------------------------------------------------------------------------------------
% Ende der Einstellungen
%---------------------------------------------------------------------------------------------------------------------------------------------------------------

%---------------------------------------------------------------------------------------------------------------------------------------------------------------
%Ab hier gibt es Inhalt
%---------------------------------------------------------------------------------------------------------------------------------------------------------------
\begin{document}

\title{Kugelfallviskosimeter}
\author{Henrik Jäger \\ 3114168 \and Lena Majer \\ 3115808}
\subtitle{S20a \\  Assistent: Patrik Zielinski}
\subject{Physikalisches Praktikum I}
\publishers{Universität Stuttgart}
\date{12. Januar 2017}
%\thanks{Assistent: Sascha Kolatschek}

\maketitle% Titelei

\tableofcontents   %Inhaltsverzeichnis
\pagebreak
\chapter{Einleitung}
\section{Ziel}
In diesem Versuch geht es um die Bestimmung der Kennlinien mehrerer Dioden. Dazu wird der Spannungsabfall  sowie die Stromstäre der einzelnen Dioden betrachtet.
\section{Grundlagen}
Nicht alle Materialien leiten gleich gut Strom. Wieso dies der Fall ist lässt sich anhand des Bändermodells erklären. In diesem Modell gibt es zwei Bänder. Das Valenzband, in dem sich die Elektronen in einem niedrigen Energieniveau befinden und das Leiterband in dem sich die Elektronen in einem höheren Energieniveau befinden. Zwischen diesen beiden Bändern kann eine Lücke entstehen. Diese wird Bandlücke genannt.\\
 Betrachtet man im Vergleich einen Isolator, so sind Valenzband und Leiterband sehr weit voneinander entfernt. Die Bandlücke ist zu groß, als dass die Elektronen wechseln könnten. Auch nach Anregung schaffen es die Elektronen nicht in das Leiterband zu wechseln. Der Stoff bleibt nicht leitend.\\
Bei Metallen gibt es verschiedene Fälle. So kann zum Beispiel das Leiterband nicht vollgefüllt sein. Da die Bandlücke jedoch so gering ist, können die Elektronen trotzdem in das Leiterband wechseln. Zudem kann das Leiterband auch voll gefüllt sein, so dass die Elektronen ohne Probleme vom einen Band zum anderen wechseln können. Als letztes kann es auch passieren, dass sich das Leitungsband und das Valenzband überlappen. Auch hier ist ein Wechsel der Elektronen von einem Band in das andere gut möglich. In diesen Fällen handelt es sich um leitende Materialien.\\ 
Es gibt sogenannte Halbleiter, bei denen das Valenzband und das Elektronenband einen gewissen Abstand voneinander haben. Durch den Abstand können die meisten Elektronen im Normalzustand nicht vom Valenzband in das Leiterband wechseln. Durch Anregung (z.B. Wärme) schaffen es deutlich mehr Elektronen zu wechseln und das Material beginnt zu leiten.\\
In einen Halbleiter können(wie bei prinzipiell allen Festkörpern) Fremdatome eingefügt werden. Diese können Donatoren sein. Das bedeutet, dass es nach Zuführung dieser Atome zu einem Elektronenüberschuss kommt. Somit können überschüssige Elektronen in das Leiterbade wechseln (n-Halbleiter). \\
Es können zu einem Halbleiter jedoch auch Elektronenfänger zugeführt werden, sogenannte Aktzeptoren. Hierbei kommt es zu Elektronenfehlstellen, da nun nur wenige Elektronen zur Verfügung stehen. Nachbaratome können in diesem Fall versuchen die Lücken durch Elektronen auszufüllen (p-Halbleiter).\\
Einen pn-Übergang erhält man durch zusammenbringen von einem n-Halbleiter und einem p-Halbleiter.\\
Dioden haben eine Sperrrichtung und eine Durchlassrichtung. Wird der n-Leiter beispielsweise an einem negativen Pol angeschlossen und der p-Halbleiter an einen positiven Pol so steigt die Leitfähigkeit nach anlegen einer Spannung, da der Diffusionsvorgang aufrechterhalten wird.\\
Wird die Diode jedoch anders herum angeschlossen und eine Spannung angelegt sinkt die Leitfähigkeit. Diese Richtung wird Sperrrichtung genannt. In dieser Richtung  ist ein Sperrstrom vorhanden.\\
Der pn-Übergang wird durch:\\
\begin{equation}
I=I_s\cdot \left[exp\left(\frac{e\cdot U}{K_B\cdot T}\right)-1\right]
\end{equation}
beschrieben.  In dieser Formel steht I für den Strom, U für die Spannung, $I_s$ für den Sperrstrom und e für die Elementarladung. T beschreibt die absolute Temperatur.\\

\section{Fragen}
\subsection{Funktion der Widerstände}
Der Widerstand in Serie zu den Dioden schützt die Dioden im Durchlassbetrieb, da die Schaltung sonst durch den geringen Widerstand der Diode kurzgeschlossen wäre. Der parallel geschaltete Widerstand verhindert in der Sperrrichtung die Überlastung der Dioden, indem er den meisten Strom ableitet.

\subsection{Wechselspannungsschaltung}
Da Channel 2 am $1k\Omega$ Widerstand liegt, kann über diesen indirekt der Stromfluss durch den Widerstand, und damit auch durch die Diode gemessen werden. An Channel 1 wird stattdessen das Ansteigen der Spannung unabhängig von der Diode gemessen.
   \pagebreak

\chapter{Messprinzip mit Skizze und Versuchsablauf}
In diesem Versuch befindet sich Öl in einem Zylinder. Der Zylinder ist mit einer Kühler- flüssigkeit umgeben. Die Temperatur des Öles wird eingestellt. Nachdem die Temperatur 5 min konstant ist, wird eine Kugel durch das Öl fallen gelassen. Die Zeit, welche die Kugel für eine Strecke l benötigt wird gestoppt. Dies wird vier Mal wiederholt. Anschließend wird das Selbe bei vier weiteren Temperaturen wiederholt. Es wird von einer laminaren Strömung ausgegangen.
\begin{center}
	\includegraphics[scale=0.27]{aufbau.jpeg}
	\captionof{figure}[]{Versuchsaufbau}
\end{center}

	
\pagebreak

	\chapter{Formeln}
Nach Höppler ist die Fallröhre kaum größer als die Kugel. Hierfür gilt:\\
\begin{equation}
\eta= K\cdot(\rho_k - \rho_{Fl})\cdot t 
\end{equation}
K entspricht einer chiastischer Konstante für die Kugel.
Die Arrhenius-Andrade-Beziehung beziehung ist gegeben durch: \\
\begin{equation}
\eta =c_1 \exp (c_2 \cdot T^{-1})
\end{equation}
$c_1$ und $c_2$ stellen Materialkonstanten dar.
	

\pagebreak
	
	\chapter{Auswertung}
	\section{Messung der Periodendauern}
    	Es wurden die Periodendauern $T_{gl}$ und $T_{gg}$ von gleichphasiger und gegenphasiger Schwingung, sowie beim Schwebungsfall die Periodendauern der Schwebung $T_S$ und der einzelnen Schwingung $T_{II}$ bei verschiedenen Längen der Aufhängung der Kopplungsmasse gemessen.
        
       	Aus den Messwerten wurde jeweils der Mittelwert gebildet.
        Für $\bar{T}_{gl 20cm}$:
        \begin{align*}
        	\bar{T}_{gl 20cm} = \frac{17,2s + 17,2s + 17,3s}{10 \cdot 3} = 1,72s
        \end{align*}
        Analog für die restlichen Werte.
        \begin{center}
        	\begin{tabular}{c||c|c|c|c}
            	$l~[cm] $ & $\bar{T}_{gl}~[s]$ & $\bar{T}_{gg}~[s]$ &	$\bar{T}_{s}~[s]$ & $\bar{T}_{0II}~[s]$ \\ \hline \hline
        		20	&1,72	&1,71	&128,74	&1,71 \\
				35	&1,71	&1,66	&44,49	&1,70 \\
				50	&1,72	&1,60	&23,62	&1,66 \\
				65	&1,72	&1,55	&15,19	&1,71 \\
        	\end{tabular}
            \captionof{table}[]{Werte}
        \end{center}
        Im Folgenden werden nur noch die Mittelwerte benutzt und somit auf die Kennzeichnung verzichtet.
        \section{Berechnung des Kopplungsgrades}
        	Hierfür kann Formel (3.1) und Formel (3.2) genutzt werden.
            Für den ersten Wert:
            \begin{align*}
    			K_1 &=\frac{T^2_{gl}-T^2_{geg}}{T^2_{gl}+T^2_{geg}} \\
            	&= \frac{(1,7233s)^2-(1,705s)^2}{(1,7233s)^2+(1,705s)^2} = 0,0107
    		\end{align*}
    		\begin{align*}
    			K_2 &=4 \cdot \frac{T_S T_{II}}{4T^2_{S}+T^2_{II}} \\
                &= 4 \cdot \frac{128,7433s \cdot 1,71s}{4(128,7433s)^2+(1,71s)^2} = 0,0133
    	\end{align*}
            Analog für die restlichen Werte:
            \begin{center}
        	\begin{tabular}{c|c|c}
            	$l~[cm] $ & $K_1~[1]$ & $K_2~[1]$ \\ \hline \hline
				20 &0,0107	&0,0133 \\
				35 &0,0326	&0,0381 \\
				50 &0,0686	&0,0701 \\
				65 &0,1077	&0,1124 \\
        	\end{tabular}
            \captionof{table}[]{Werte}
        \end{center}
        
Die Formel für $K_2$ ist die genauere der beiden, weil hier nur Quadrate im Nenner stehen, und so auch die Fehler nur im Nenner quadratisch eingehen. Dies sieht man auch in der Fehlerfortpflanzung.
        \section{$x(t)$-Diagramm}
    	\begin{center}
   			\begin{gnuplot}[terminal=pdf,terminaloptions={font ",10" linewidth 1},scale=1.2]
            set xrange [0:20]
            set ylabel "Horizontale Auslenkung (mit offset)"
            set xlabel "Zeit t [s]"
            set key box opaque bottom
                plot 'Daten/schwebung.csv' using ($1):(($2+3000)/1000) title 'P1 Schwebung' w l \ 
                ,'Daten/schwebung.csv' using ($1):(($3+1000)/1000) title 'P2 Schwebung' w l \
                , 'Daten/gleich.csv' using ($1):(($2+2200)/1000) title 'P1 gleichphasig' w l \
                , 'Daten/gleich.csv' using ($1):(($3-800)/1000) title 'P2 gleichphasig' w l \
                ,'Daten/gegen.csv' using ($1):(($2+900)/1000) title 'P1 gegenphasig' w l \ 
                ,'Daten/gegen.csv' using ($1):(($3-1900)/1000) title 'P2 gegenphasig' w l 
			\end{gnuplot}
			\captionof{figure}[]{Vergleich der Fundamentalschwingungen}
   		\end{center}
        Die gleich- und gegenphasige Schwingung haben im idealen Fall eine kontante Amplitude. Die Amplitude der Schwebung wird von einer  Sinus-Funktion eingehüllt.
            
	\pagebreak
    \chapter{Fehlerrechnung}
	Um das Ventilvolumen in die Formel einzubringen, wird $\gamma$ in Abhängigkeit der Länge $l$, statt der Fläche $A$ und dem Volumen $V$ geschrieben.
      \begin{align*}
     		\gamma = \frac{16 \pi \cdot m \cdot l \cdot f_0^2}{d^2 \cdot p_0}
      \end{align*}
      \begin{align*}
     		\Delta\gamma_{\text{Luft}} &= 
            \left| \frac{16\pi l f_0^2}{d^2 p_0}\right| \Delta m + 
            \left| \frac{16\pi m f_0^2}{d^2 p_0}\right| \Delta l + 
            \left| \frac{32\pi m l f_0}{d^2 p_0}\right| \Delta f_0 + 
            \left| -\frac{32 \pi m l f_0^2}{d^3 p_0}\right| \Delta d \\&+ 
            \left| -\frac{16 \pi m l f_0^2}{d^2 p_0^2}\right| \Delta p_0 \\&=
            \left| \frac{16\pi (0,45m + 0,001m) (13,4Hz)^2}{(0,016m)^2 (103100Pa)}\right| (0,0000001kg)\\ &+ 
            \left| \frac{16\pi (0,0088781kg) (13,4Hz)^2}{(0,016m)^2 (103100Pa)}\right| (0,001m) \\ &+ 
            \left| \frac{32\pi (0,0088781kg) (0,45m + 0,001m) (13,4Hz)}{(0,016m)^2 (103100Pa)}\right| (0,2Hz) \\ &+ 
            \left| -\frac{32 \pi (0,0088781kg) (0,45m + 0,001m) (13,4Hz)^2}{(0,016m)^3 (103100Pa)}\right| (0,00001m)\\&+ 
            \left| -\frac{16 \pi (0,0088781kg) (0,45m + 0,001m) (13,4Hz)^2}{(0,016m)^2 (103100Pa)^2}\right| (100Pa) \\ &= 0,048
      \end{align*} \\
      
      Analog für die weiteren Fehler
      \begin{center}
      \begin{tabular}{c|c}
$l ~[\text{m}]$ & $\Delta\gamma ~[1]$    \\ \hline
0,45         & 0,048 \\
0,40          & 0,045\\
0,35         & 0,045 \\
0,30          & 0,042 \\
0,25         & 0,039 \\
0,20          & 0,037 \\
0,15         & 0,036 \\
0,10          & 0,035	
		\end{tabular}
	\end{center}
      
\captionof{table}[]{Fehler für die Messung mit Luft als Kolbenfüllung.}
\begin{center}
      \begin{tabular}{c|c}
$l ~[\text{m}]$ & $\Delta\gamma ~[1]$    \\ \hline
0,45 & 0,048 \\
0,40  & 0,045 \\
0,35 & 0,042 \\
0,30  & 0,040 \\
0,25 & 0,038 \\
0,20  & 0,036 \\
0,15 & 0,035 \\
0,10  & 0,034
		\end{tabular}
	\end{center}
      
\captionof{table}[]{Fehler für die Messung mit Kohlenstoffdioxid als Kolbenfüllung.}

\begin{center}
      \begin{tabular}{c|c}
$l ~ [\text{m}]$ & $\Delta\gamma ~[1]$    \\ \hline
0,45   & 0,053 \\
0,40    & 0,050 \\
0,35   & 0,048 \\
0,30    & 0,045 \\
0,24   & 0,042 \\
0,19   & 0,038 \\
0,14   & 0,047 \\
0,095  & 0,038 
		\end{tabular}
	\end{center}
      
\captionof{table}[]{Fehler für die Messung mit Argon als Kolbenfüllung.}
\ \\
Es soll nur ein Wert für den Fehler angegeben werden. Anstatt wie bei den Messwerten zu mitteln, wird hier jedoch der größte Fehler angenommen, damit der Fehlerwert nicht zu gering ausfällt.

\begin{align*}
 	\Delta \gamma_{\text{Luft}} & = 0,048\\
    \Delta \gamma_{\text{CO}_2} & = 0,048\\
    \Delta \gamma_{\text{Argon}}& = 0,053
\end{align*}

     	\section{Mögliche Abweichungen}
        	Im Versuch wird angenommen, dass die Erregerfrequenz beim Maximum der Schwingung der Eigenfrequenz des ungedämpften Systems entspricht. Durch Dämpfung fällt die Frequenz jedoch geringer aus, was sich durch den Zusammenhang $\gamma \propto f^2$ stark auf das Ergebnis auswirkt. Der Wert für den gemessenen Adiabatenexponent liegt also unter dem tatsächlichen. Außerdem wird nicht mit einem isolierten System gemessen, sodass der Vorgang nicht vollständig adiabatisch verläuft. Reibung verursacht ebenfalls einen Fehler.
	\pagebreak
    
	\chapter{Zusammenfassung}
In diesem Versuch konnte die Arrhenius-Andrade-Beziehung der expotentiellen Temperaturabhänigkeit der Viskosität bestätigt werden. \\
\\
Als Wert der stoffabhängigen Konstanten erhielt man einen Wert von
\begin{center}
\begin{tabular}{c|c}
Konstante & Wert \\ \hline
$c_1$ & $(3,5 \cdot 10^{-9} \pm 0,33 \cdot 10^{-9}) Pa \cdot s $ \\
$c_2$ &  $(5249,64 \pm 131,17) K $\\
\end{tabular}
\end{center}

Die Viskositäten in Abhängigkeit der Temperatur:
\begin{center}
   \begin{tabular}{c|c|c}
  		Temperatur $T [\si{\kelvin}]$ & Fallzeit $t [\si{\second}]$ & Viskosität $\eta [\si{\pascal\second}]$ \\ \hline \hline
        298,25 & 38,784	& 0,1538 \\
		308,25 & 20,994	& 0,0832 \\
		318,25 & 13,138	& 0,0521 \\
		328,25 & 7,806	& 0,0310 \\
		333,25 & 6,336	& 0,0251 \\
   \end{tabular}
   \captionof{table}[]{Werte der dynamischen Viskosität}
   \end{center}
   Dabei ist der Fehler jeweils $\Delta \eta = 0,002 Pas$
   

\pagebreak
	\chapter{Anhang}
    \includegraphics[scale=0.6]{protokoll.pdf}
    \captionof{figure}[]{Messprotokoll}
	










\end{document}
