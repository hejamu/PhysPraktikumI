\chapter{Auswertung}
	Die dynamische Viskosität wird nach Formel (3.1) berechnet.
    \begin{align*}
    	\eta = K (\rho_k - \rho_{Fl})t 
    \end{align*}
    
    Beispielhaft für die Temperatur $T_1 = 298.25 \si{\kelvin}$:
    \begin{align*}
    	\eta = 0.0005445 \si{\pascal\centi\metre^3\per\gram}\cdot (8.141 \si{\gram\per\centi\metre^3} - 0.86 \si{\gram\per\centi\metre^3})\cdot 38.784 \si{\second} = 0.1538 \si{\pascal\second}
    \end{align*}
   Tabelle 4.1 enthält alle errechneten Werte. 
   \begin{center}
   \begin{tabular}{c|c|c}
  		Temperatur $T [\si{\kelvin}]$ & Fallzeit $t [\si{\second}]$ & Viskosität $\eta [\si{\pascal\second}]$ \\ \hline \hline
        298,25 & 38,784	& 0,1538 \\
		308,25 & 20,994	& 0,0832 \\
		318,25 & 13,138	& 0,0521 \\
		328,25 & 7,806	& 0,0310 \\
		333,25 & 6,336	& 0,0251 \\
   \end{tabular}
   \captionof{table}[]{Werte der dynamischen Viskosität}
   \end{center}
   
   Zur Bestimmung der Konstanten des Exponentialgesetzes (3.2) wird $\eta$ semilogarithmisch über $1/T$ aufgetragen. 
   \begin{center}
   \begin{gnuplot}[terminal=pdf,terminaloptions={font ",10" linewidth 1},scale=1.2]
		set xlabel "Temperatur [1/T]"
		set ylabel "Viskosität [Eta]"
        set logscale y
        set yrange [0.02:0.2]
        set fit errorvariables
        set key left
        
        f(x) = a*exp(b*x)
        fit f(x) 'data.csv' using (1/$1):($2) via a,b
        a_err = a_err 
        b_err = b_err
        g(x) = (a+0.000000000002)*exp((b+20)*x)
        h(x) = (a-0.000000000002)*exp((b-20)*x)
        set label sprintf("c_1 = %.12f", a) at 0.0032,0.06
		set label sprintf("c_2 = %.12f", b) at 0.0032,0.05
        set label sprintf("dc_1 = %.12f", a_err) at 0.0032,0.04
		set label sprintf("dc_2 = %.12f", b_err) at 0.0032,0.03
        plot 'data.csv' using (1/$1):($2):($3):($4) t "Messwerte" with xyerrorbars,f(x),g(x) t "f_min",h(x) t "f_max"
\end{gnuplot}
\captionof{figure}[]{Graphische Darstellung der Arrhenius-Andrade-Beziehung}
   \end{center}
 Mithilfe von Gnuplot wurde eine Fitfunktion der Form 
 \begin{align*}
 	f(x) = c_1 \exp(c_2 x^{-1})
 \end{align*}
 angefittet.
 Dabei ergaben sich die Werte von $c_2 = 5249,6449 \si{\kelvin} $ und  $c_1=0.000000003466 Pas$. \\
 


   
  