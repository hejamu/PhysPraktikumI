\chapter{Einleitung}
\section{Ziel}
Bei diesem Versuch soll die Viskosität von Pumpenöl bestimmt werden. Nach Höppler wird die Viskosität mit Hilfe eines Kugelfallviskosmeters und bei variierter Temperatur bestimmt.
\section{Grundlagen}
Viskosität beschreibt die innere Reibung - die Zähigkeit – von Fluiden. Die Viskosität bei Gasen steigt mit zunehmender Temperatur. Bei Flüssigkeiten ist es genau anderst herum. Bei steigender Temperatur singt die Viskosität. Dies wird durch die Arrhenius- Andrade-Beziehung beschrieben: \\
\begin{equation}
\eta =c_1 \exp (c_2\cdot T{-1})
\end{equation}
$c_1$ und $c_2$ stellen Materialkonstanten dar. $c_2$ ist mit einer Aktivierungsenergie vergleichbar. Die absolute Temperatur T muss in Kelvin angegeben werden.\\
Sobald Gas schichten oder Flüssichkeitsschichten parallel zueinander bewegen, kommt es zu Reibungen, die großen Einfluss auf die Fortbewegung haben und zu Erniedrigung des Geschwindigkeitsgefälle führt.\\
Wird beispielweise eine Platte durch eine Flüssigkeit gezogen, muss die Flüssigkeit sich vor der Platte scheren. Hierbei kommt es zu zwischenmolekularen Kräften – den inneren Kräften. \\
Für eine laminare Strömung gilt nach Newton: \\
\begin{equation}
F_r = \eta \cdot A \cdot \frac{\partial v}{\partial x}
\end{equation}
Hierbei entspricht $\frac{\partial v}{\partial x}$ dem Geschwindichkeitsgefälle. A entspricht der eingetauchten Fläche.
Taucht der Begriff kinematische Viskosität auf,  versteht man damit die Definition:\\
\begin{equation}
V:=\frac{\eta}{\rho}
\end{equation}
$\rho$ entspricht der Dichte des verwendeten Fluids.\\
Als dynamische Viskosität wird mit $\eta$ bezeichnet. $\eta$ ist definiert als:\\
\begin{equation}
\eta  := \frac{\tau }{ \dot{\gamma} }
\end{equation}
Unter einer Laminaren Strömung versteht man eine Strömung ohne Verwirbelungen. Bei turbulenten Strömungen sind jedoch Verwirbelungen zu finden.\\
Markant zur Bestimmung, ob es sich um eine turbulente oder laminare Strömung handelt, ist die Reynoldszahl. Diese stellt ein Verhältnis zwischen Trägheitskräften und Zähigkeitkräften dar. Die Reynoldszahl ist definiert als:\\
Re = Beschleunigungsarbeit/ Reibungsarbeit \\
Bei Werten der Reynoldszahl $> Re_{krit}$ liegt eine Turbulente Strömung vor. Sind die Werte der Reynoldszahl $< Re_{krit}$  handelt es sich um eine laminare Strömung.\\
Bei laminarer Strömung kann das Gesetz von Hagen- Poiseulle verwendet werden. \\
\begin{equation}
I = \frac{\pi \cdot R^4}{8\cdot L\cdot\eta} \cdot \Delta p
\end{equation}
Um die Viskosität zu bestimmen gibt es zwei verschiedenen Vorgehensweisen nach Stokes oder nach Höppler. Nach Stokes wird die Auftriebskraft und die Stoksche Reibkraft sowie die Gewichtskraft betrachtet. Die Summe aller Kräfte muss nach Newton 0 ergeben.\\
Nach Höppler ist die Fallröhre kaum größer als die Kugel. Hierfür gilt:\\
\begin{equation}
\eta= K\cdot ( \varrho_k - \varrho_{Fl})\cdot t 
\end{equation}
K entspricht einer chiastischer Konstante für die Kugel.




\pagebreak