\chapter{Fehlerrechnung}
	Da die Spannung über das Oszilloskop gemessen wird und wir die Skalierung immer nachjustieren kann man einenen geringen Fehler von 
\begin{align*}
	\Delta U = 0,05 V
\end{align*}
    annehmen.
    Die Frequenz kann ebenfalls ziemlich genau aus dem Oszilloskop bestimmt werden, man nehmen an 
    \begin{align*}
    	\Delta f = 0,5 Hz
    \end{align*}
	\section{Fehler für den Widerstand $R_{sp}$ über Fehlerfortpflanzung}
    	Der Widerstand $R_{sp}$ wird über Formel
        \begin{equation}
        	R_{sp} = R \cdot \left( \frac{U_e}{U_a} - 1 \right)
        \end{equation}
        bestimmt.
        Es gilt:
        \begin{align*}
        	\Delta R_{sp} &= \left|\frac{\partial R_{sp}}{\partial U_a}\right|  \cdot \Delta U_a\\
            &=\left|\frac{R\cdot U_e}{U_a^2}\right|\cdot\Delta U_a\\
			&=\left|\frac{1\Omega\cdot 1 V}{(0,7365V)^2}\right|\cdot0,05 V\\
            &=0,0922\Omega
        \end{align*}
    	
\section{Fehler für die Stärke des B-Felds über Fehlerfortpflanzung}
	
	Die Magnetfeldstärke wird über die Formel \begin{equation}
		B =\frac{1}{2 \pi f \cdot n\cdot A}U_{ind}
	\end{equation}
    bestimmt.\\
    Weil man hier in der mV-Skala gemessen hat, ist das Oszilloskop auch genauer, sodass man jetzt $\Delta U = 0,005V$ annehmen kann.
	Es gilt: 
    \begin{align*}
    	\Delta B &= \left| \frac{\partial B}{\partial f} \right| \Delta f + \left| \frac{\partial B}{\partial U_{ind}} \right| \Delta U_{ind} \\
        &=\left|\frac{U_{ind}}{2\pi f^2 n A} \right| \Delta f + \left| \frac{1}{2\pi f n A} \right| \Delta U_{ind}\\
       &=\left|\frac{0,0832V}{2\pi (100Hz)^2 \cdot 1000 \cdot 0,00035m^2} \right| 0,5 Hz + \left| \frac{1}{2\pi 100Hz \cdot 1000 \cdot 0,00035 m^2} \right| 0,005V\\
       & = 0,04 \cdot 10^{-3} Wb
       \end{align*}



\pagebreak