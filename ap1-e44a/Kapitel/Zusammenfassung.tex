\chapter{Zusammenfassung}
    Im ersten Versuchsteil konnte mithilfe des Verhältnisses zwischen Eingangs- und Ausgangsspannung der Widerstand der großen Spule bestimmt werden.
    
   \begin{equation*}
   		R = (0,3578 \pm 0,0922) \Omega
   \end{equation*}
    Ausßerdem wurde mit der Frequenz der Spannung die Induktivität der großen Spule bestimmt, indem sie als komplexer Widerstand betrachtet wurde.
    
    \begin{equation*}
   		L_{gro\textit{ß}{}} = 0,8 \cdot 10^{-3} H
   \end{equation*}
   Im zweiten Versuchsteil wurde das magnetische Feld im Inneren der großen Spule untersucht. Das Magnetfeld zeigt einen relativ konstanten Wert innerhalb der Spule, nimmt dann aber sehr schnell am Ende ab und beschreibt dann eine exponentielle Abnahme. Am Ende der Spule hat das Magnetfeld einen Wert von 
   \begin{equation*}
   		B = (0,29 \pm 0,04)\cdot 10^{-3} Wb
   \end{equation*}
   \\
    Für die Frequenzabhängigkeit der Induktionsspannung kam in Versuchsteil 3 heraus, dass es ein linearer Zusammenhang ist.
    \\
    Am Ende wurden die Formen verschiedener Eingangssignale über das Oszilloskop betrachtet und die Rolle der Induktionsspannung als Ableitung der Erregerspannung diskutiert. Letztlich wurde die Selbstinduktion untersucht und gezeigt, wie eine Spule dem Anstieg des Stromes entgegenwirkt. Hierbei konnte durch die Betrachtung des Messaufbaus als Schwingkreis die Induktivität der kleinen Spule berechnet werden.\\
    \begin{equation*}
   		L_{klein} = 0,3 \cdot 10^{-6} H
   \end{equation*}
    
	\pagebreak
