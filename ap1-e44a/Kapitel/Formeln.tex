\chapter{Formeln}
    Zur Bestimmung der Stärke des magnetischen Feldes B wird die Formel:
    \begin{equation}
		B =\frac{1}{2 \pi f \cdot n\cdot A}U_{ind}
	\end{equation}
verwendet. Hierbei beschreibt $f$ die Frequenz der Wechselspannung, $n$ die Windungszahl und A die Spulenfläche.\\
\\
Sobald eine zeitliche Änderung des Magnetischen Fusses in einer Spule vorgenommen wird entsteht eine Induktionsspannung. Hierfür gilt:\\
\begin{equation}
	\Phi=B \cdot A
\end{equation}
	
     Der ohmschen Widerstand wird über den spezifischen Widerstand $\rho_{cu}$ berechnet:\\
  \begin{equation}
    	R_{sp} =  \rho_{cu} \cdot \frac{l_{sp}}{A_{sp}}
        \label{}
  \end{equation}
  l beschreibt hierbei die Länge der Spule und A die Fläche der Spule.\\
  
  Mit $l = \pi \cdot d \cdot n_a$ und $A = \pi \cdot \frac{1}{4} d^2$ ist der Widerstand der Spule gegeben:
    \begin{equation}
    	R_{sp}=\rho\cdot\frac{\pi\cdot 2 r\cdot n_a}{\pi\cdot\frac{1}{4}\cdot d^2} = \rho\cdot\frac{8\cdot r\cdot n_a}{d^2} \\
        \label{spezifischerWiderstand}
    \end{equation}
    \\
    Die Induktivität L kann über die Spannungen bestimmt werden. Hierfür gilt:
    \begin{equation}
    L= \sqrt{\frac{U_e \cdot R}{U_a \cdot \omega^2}}
    \label{InduktivitaetSpannung}
    \end{equation}
  \\
  Mit der Formel
\begin{equation}
	f  = \frac{1}{2\pi\cdot\sqrt{ L\cdot C}}
\end{equation}
gilt für die Induktivität L der Spule:
\begin{equation}
	  L= \frac{1}{f^2 \cdot 4\pi^2 \cdot C} 
      \label{InduktivitaetFrequenz}
\end{equation}
 
        \pagebreak
    