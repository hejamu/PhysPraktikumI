\chapter{Einleitung}
\section{Ziel}
Das Ziel dieses Versuches ist es die Eigenschaften von Spulen in Abhänigkeit von der Frequenz, der Feldstärke und der Eingangsspannung sowie der  Ausgangsspannung zu untersuchen. Ebenso wird die Lage der Spule und die damit induzierte Spannung untersucht (Selbstinduktionsverhalten).
\section{Grundlagen}
Befindet sich ein bewegtes Elektron im Einfluss eines Magnetfeldes, wird das Elektron senkrecht zur Bewegungsrichtung und Senkrecht zum Magnetfeld abgelenkt. Das Elektron beschreibt in diesem Fall eine Kreisbahn.\\
Um elektromagnetische Eigenschaften zu beschreiben verwendet man einige Formeln:\\
\begin{equation}
\oint_c H \,ds= \int_A \vec j \,dA
\end{equation}
\begin{equation}
Div \vec D = \varrho 
\end{equation}
\\(Gaußscher Satz)

\begin{equation}
Div \vec B = 0 
\end{equation}
(geschlossene B-Feldinien)
 \begin{equation}
 Rot \vec E = - \frac{d\vec B}{dt} 
 \end{equation}
 (Faradaysches Induktionsgesetz)
 \begin{equation}
 Rot \vec H = \vec j + \frac{d \vec D}{dt} 
 \end{equation}
Hierbei steht $\vec j$ für die elektrische Flussdichte, $\vec H$ für die magnetische Feldstärke und $\vec D$ für die dielektrische Verschiebung für die gilt:\\
\begin{equation}
\vec D = \varepsilon\cdot\varepsilon_0 \cdot\vec E 
\end{equation}
\begin{equation}
\vec H=n\cdot\frac{\vec I}{l}
\end{equation}
beschreibt das Durchflussgesetz. Hiermit wird klar, dass eine stromdurchflossene Spule der Länge l und mit der Windungszahl n ein Magnetfeld erzeugt.\\
Wenn das Magnetfeld eines stromdurchflossenen Leiters bestimmt werden soll, wird Biot-Savar verwendet:\\
\begin{equation}
D \vec H= \frac{I}{4 \pi \cdot r^3} \cdot d \vec l \times \vec r
\end{equation}
Die Feldstärke und Flussdichte wird durch eine Permeabilitätszahl und die magnetische Feldkonstante verknüpft. \\
\begin{equation}
\vec B = \mu_r \cdot \mu_0 \cdot \vec H
\end{equation}
Sobald eine zeitliche Änderung des magnetischen Flusses in einer Spule vorgenommen wird entsteht eine Induktionsspannung. Hierfür gilt:\\
\begin{equation}
\Phi=B \cdot A
\end{equation}

\begin{equation}
U_{ind}= -n\cdot \frac{d\Phi}{dt} = -n \cdot \dot{\Phi}
\end{equation}
\pagebreak