\chapter{Einleitung}
\section{Ziel}
In diesem Versuch geht es um die Bestimmung der Kennlinien mehrerer Dioden. Dazu wird der Spannungsabfall  sowie die Stromstäre der einzelnen Dioden betrachtet.
\section{Grundlagen}
Nicht alle Materialien leiten gleich gut Strom. Wieso dies der Fall ist lässt sich anhand des Bändermodells erklären. In diesem Modell gibt es zwei Bänder. Das Valenzband, in dem sich die Elektronen in einem niedrigen Energieniveau befinden und das Leiterband in dem sich die Elektronen in einem höheren Energieniveau befinden. Zwischen diesen beiden Bändern kann eine Lücke entstehen. Diese wird Bandlücke genannt.\\
 Betrachtet man im Vergleich einen Isolator, so sind Valenzband und Leiterband sehr weit voneinander entfernt. Die Bandlücke ist zu groß, als dass die Elektronen wechseln könnten. Auch nach Anregung schaffen es die Elektronen nicht in das Leiterband zu wechseln. Der Stoff bleibt nicht leitend.\\
Bei Metallen gibt es verschiedene Fälle. So kann zum Beispiel das Leiterband nicht vollgefüllt sein. Da die Bandlücke jedoch so gering ist, können die Elektronen trotzdem in das Leiterband wechseln. Zudem kann das Leiterband auch voll gefüllt sein, so dass die Elektronen ohne Probleme vom einen Band zum anderen wechseln können. Als letztes kann es auch passieren, dass sich das Leitungsband und das Valenzband überlappen. Auch hier ist ein Wechsel der Elektronen von einem Band in das andere gut möglich. In diesen Fällen handelt es sich um leitende Materialien.\\ 
Es gibt sogenannte Halbleiter, bei denen das Valenzband und das Elektronenband einen gewissen Abstand voneinander haben. Durch den Abstand können die meisten Elektronen im Normalzustand nicht vom Valenzband in das Leiterband wechseln. Durch Anregung (z.B. Wärme) schaffen es deutlich mehr Elektronen zu wechseln und das Material beginnt zu leiten.\\
In einen Halbleiter können(wie bei prinzipiell allen Festkörpern) Fremdatome eingefügt werden. Diese können Donatoren sein. Das bedeutet, dass es nach Zuführung dieser Atome zu einem Elektronenüberschuss kommt. Somit können überschüssige Elektronen in das Leiterbade wechseln (n-Halbleiter). \\
Es können zu einem Halbleiter jedoch auch Elektronenfänger zugeführt werden, sogenannte Aktzeptoren. Hierbei kommt es zu Elektronenfehlstellen, da nun nur wenige Elektronen zur Verfügung stehen. Nachbaratome können in diesem Fall versuchen die Lücken durch Elektronen auszufüllen (p-Halbleiter).\\
Einen pn-Übergang erhält man durch zusammenbringen von einem n-Halbleiter und einem p-Halbleiter.\\
Dioden haben eine Sperrrichtung und eine Durchlassrichtung. Wird der n-Leiter beispielsweise an einem negativen Pol angeschlossen und der p-Halbleiter an einen positiven Pol so steigt die Leitfähigkeit nach anlegen einer Spannung, da der Diffusionsvorgang aufrechterhalten wird.\\
Wird die Diode jedoch anders herum angeschlossen und eine Spannung angelegt sinkt die Leitfähigkeit. Diese Richtung wird Sperrrichtung genannt. In dieser Richtung  ist ein Sperrstrom vorhanden.\\
Der pn-Übergang wird durch:\\
\begin{equation}
I=I_s\cdot \left[exp\left(\frac{e\cdot U}{K_B\cdot T}\right)-1\right]
\end{equation}
beschrieben.  In dieser Formel steht I für den Strom, U für die Spannung, $I_s$ für den Sperrstrom und e für die Elementarladung. T beschreibt die absolute Temperatur.\\

\section{Fragen}
\subsection{Funktion der Widerstände}
Der Widerstand in Serie zu den Dioden schützt die Dioden im Durchlassbetrieb, da die Schaltung sonst durch den geringen Widerstand der Diode kurzgeschlossen wäre. Der parallel geschaltete Widerstand verhindert in der Sperrrichtung die Überlastung der Dioden, indem er den meisten Strom ableitet.

\subsection{Wechselspannungsschaltung}
Da Channel 2 am $1k\Omega$ Widerstand liegt, kann über diesen indirekt der Stromfluss durch den Widerstand, und damit auch durch die Diode gemessen werden. An Channel 1 wird stattdessen das Ansteigen der Spannung unabhängig von der Diode gemessen.
   \pagebreak