\chapter{Formeln}
    
Der pn-Übergang wird durch:\\
    \begin{equation}
		I=I_s \cdot \left[exp \left(\frac{e\cdot U}{k_B*T} \right) - 1 \right]
        \label{Sperrstrom}
	\end{equation}
beschrieben.  In dieser Formel steht I für den Strom, U für die Spannung, $I_s$ für den Sperrstrom und e für die Elementarladung. T beschreibt die Absolute Temperatur.  \\

  Formel \eqref{Sperrstrom} ist für große Ströme vereinfacht auch 
  \begin{equation}
  	I=I_s \cdot exp \left(\frac{e\cdot U}{k_B*T} \right)
    \label{Sperrstrom-vereinfacht}
  \end{equation}
  weil für $T = 293,15 K$ und $U = 0,1 V$ 
  \begin{equation}
  		 \exp\left(\frac{e\cdot U}{k_B \cdot T} \right) \approx 55
  \end{equation}
  und somit wird die $-1$ vernachlässigbar. \\
  Die Energie, die von einer Leuchttdiode in Form von Licht emmitiert wird kann durch folgende Formel beschrieben werden:  \\ 
  
    \begin{equation}
		eU=hv=\frac{hc}{\lambda}
	\end{equation}
Hierbei beschreibt $h$ das plancksche Wirkungsquantum und c die Lichtgeschwindigkeit.
\\
Umgeschrieben kann die  Wellenlänge $\lambda$ des Lichts berschrieben werden durch:\\

    \begin{equation}
\lambda= \frac{hc}{eU}
\end{equation}


	\pagebreak