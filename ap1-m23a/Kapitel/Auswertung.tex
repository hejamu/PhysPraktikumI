\chapter{Auswertung}
	\section{Messung der Periodendauern}
    	Es wurden die Periodendauern $T_{gl}$ und $T_{gg}$ von gleichphasiger und gegenphasiger Schwingung, sowie beim Schwebungsfall die Periodendauern der Schwebung $T_S$ und der einzelnen Schwingung $T_{II}$ bei verschiedenen Längen der Aufhängung der Kopplungsmasse gemessen.
        
       	Aus den Messwerten wurde jeweils der Mittelwert gebildet.
        Für $\bar{T}_{gl 20cm}$:
        \begin{align*}
        	\bar{T}_{gl 20cm} = \frac{17,2s + 17,2s + 17,3s}{10 \cdot 3} = 1,72s
        \end{align*}
        Analog für die restlichen Werte.
        \begin{center}
        	\begin{tabular}{c||c|c|c|c}
            	$l~[cm] $ & $\bar{T}_{gl}~[s]$ & $\bar{T}_{gg}~[s]$ &	$\bar{T}_{s}~[s]$ & $\bar{T}_{0II}~[s]$ \\ \hline \hline
        		20	&1,72	&1,71	&128,74	&1,71 \\
				35	&1,71	&1,66	&44,49	&1,70 \\
				50	&1,72	&1,60	&23,62	&1,66 \\
				65	&1,72	&1,55	&15,19	&1,71 \\
        	\end{tabular}
            \captionof{table}[]{Werte}
        \end{center}
        Im Folgenden werden nur noch die Mittelwerte benutzt und somit auf die Kennzeichnung verzichtet.
        \section{Berechnung des Kopplungsgrades}
        	Hierfür kann Formel (3.1) und Formel (3.2) genutzt werden.
            Für den ersten Wert:
            \begin{align*}
    			K_1 &=\frac{T^2_{gl}-T^2_{geg}}{T^2_{gl}+T^2_{geg}} \\
            	&= \frac{(1,7233s)^2-(1,705s)^2}{(1,7233s)^2+(1,705s)^2} = 0,0107
    		\end{align*}
    		\begin{align*}
    			K_2 &=4 \cdot \frac{T_S T_{II}}{4T^2_{S}+T^2_{II}} \\
                &= 4 \cdot \frac{128,7433s \cdot 1,71s}{4(128,7433s)^2+(1,71s)^2} = 0,0133
    	\end{align*}
            Analog für die restlichen Werte:
            \begin{center}
        	\begin{tabular}{c|c|c}
            	$l~[cm] $ & $K_1~[1]$ & $K_2~[1]$ \\ \hline \hline
				20 &0,0107	&0,0133 \\
				35 &0,0326	&0,0381 \\
				50 &0,0686	&0,0701 \\
				65 &0,1077	&0,1124 \\
        	\end{tabular}
            \captionof{table}[]{Werte}
        \end{center}
        
Die Formel für $K_2$ ist die genauere der beiden, weil hier nur Quadrate im Nenner stehen, und so auch die Fehler nur im Nenner quadratisch eingehen. Dies sieht man auch in der Fehlerfortpflanzung.
        \section{$x(t)$-Diagramm}
    	\begin{center}
   			\begin{gnuplot}[terminal=pdf,terminaloptions={font ",10" linewidth 1},scale=1.2]
            set xrange [0:20]
            set ylabel "Horizontale Auslenkung (mit offset)"
            set xlabel "Zeit t [s]"
            set key box opaque bottom
                plot 'Daten/schwebung.csv' using ($1):(($2+3000)/1000) title 'P1 Schwebung' w l \ 
                ,'Daten/schwebung.csv' using ($1):(($3+1000)/1000) title 'P2 Schwebung' w l \
                , 'Daten/gleich.csv' using ($1):(($2+2200)/1000) title 'P1 gleichphasig' w l \
                , 'Daten/gleich.csv' using ($1):(($3-800)/1000) title 'P2 gleichphasig' w l \
                ,'Daten/gegen.csv' using ($1):(($2+900)/1000) title 'P1 gegenphasig' w l \ 
                ,'Daten/gegen.csv' using ($1):(($3-1900)/1000) title 'P2 gegenphasig' w l 
			\end{gnuplot}
			\captionof{figure}[]{Vergleich der Fundamentalschwingungen}
   		\end{center}
        Die gleich- und gegenphasige Schwingung haben im idealen Fall eine kontante Amplitude. Die Amplitude der Schwebung wird von einer  Sinus-Funktion eingehüllt.
            
	\pagebreak